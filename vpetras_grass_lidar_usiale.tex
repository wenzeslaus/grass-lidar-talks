% Using Free and Open Source Solutions in Geospatial Science Education
% This work by Vaclav Petras is licensed under
% a Creative Commons Attribution-ShareAlike 4.0 International License.

\documentclass[xcolor={dvipsnames,usenames},beamer,aspectratio=169]{beamer}
% ,handout,notes=show

\makeatletter
\def\beamer@framenotesbegin{% at beginning of slide
  \gdef\beamer@noteitems{}%
  \gdef\beamer@notes{{}}% used to be totally empty.
}
\makeatother

\usepackage{textcomp}
\usepackage[utf8]{inputenc}
\usepackage[american]{babel}
\usepackage{graphicx}
\usepackage{url}
\usepackage{amssymb}

\usepackage{tikz}
\usetikzlibrary{arrows,shapes,spy,calc}

\tikzstyle{every picture}+=[remember picture]
\tikzstyle{na} = [baseline=-.5ex]

% frames have to be fragile
\newif\ifnotes
% \input{tmpnotessettings}
% \notestrue


\ifnotes
\setbeamertemplate{note page}[plain]
% \setbeamertemplate{note page}[compress]
\setbeamerfont{note page}{size=\large}
% \setbeameroption{show only notes}
\setbeameroption{show notes}
\usepackage{pgfpages}
\pgfpagesuselayout{2 on 1}[a4paper,border shrink=5mm]%
\else
%\setbeameroption{hide notes}
\fi
%\notesfalse

\usepackage[absolute,overlay]{textpos}

\usepackage{listings}


% \usetheme{Warsaw}
\usetheme{Madrid}
% \usetheme{Frankfurt}
% \useoutertheme{infolines}
\usecolortheme[named=MidnightBlue]{structure}
% \usecolortheme[named=PineGreen]{structure}
\setbeamertemplate{navigation symbols}{}

\setbeamertemplate{itemize items}[default]
\setbeamertemplate{enumerate items}[default]
% \useinnertheme{rectangles}
\setbeamertemplate{blocks}[default]


%%%%%%%%%%%%%%%%%%%%%%%%%%%%%%%%%%%%%%%%%%%%%%%%%%%%%%%%%%%%%%%%%%%%
%%%%%%%%%%%%%%%%%%%%%%%%%%%%%%%%%%%%%%%%%%%%%%%%%%%%%%%%%%%%%%%%%%%%

% \newcommand{\n}[1]{$^{\color{gray}{\mbox{\tiny#1}}}$}
\newcommand{\n}[1]{$^{\textcolor{gray}{\mbox{\tiny #1}}}$}

%%%%%%%%%%%%%%%%%%%%%%%%%%%%%%%%%%%%%%%%%%%%%%%%%%%%%%%%%%%%%%%%%%%%%%%%%%%%%%%
\newcommand{\gmodule}[1]{\href{http://grass.osgeo.org/grass71/manuals/#1.html}{\emph{#1}}}
\newcommand{\amodule}[1]{\href{http://grass.osgeo.org/grass70/manuals/addons/#1.html}{\emph{#1}}}
\newcommand{\module}[1]{\emph{#1}}
\newcommand{\grasslink}{\href{http://grass.osgeo.org/}{GRASS GIS}}

%%%%%%%%%%%%%%%%%%%%%%%%%%%%%%%%%%%%%%%%%%%%%%%%%%%%%%%%%%%%%%%%%%%%
%%%%%%%%%%%%%%%%%%%%%%%%%%%%%%%%%%%%%%%%%%%%%%%%%%%%%%%%%%%%%%%%%%%%

\title[Processing of point clouds in GRASS GIS]
{Efficient processing of dense point clouds in GRASS GIS}
\subtitle{at US-IALE 2016 Annual Meeting (lab presentation at CGA)}
%\pdforstring{}{}

\author[Vaclav Petras]
{Vaclav Petras (Vashek)\\%\n{1,2}\\
{
\scriptsize
Douglas Newcomb, %\n{3},
%\mbox{
Helena Mitasova%\n{1,2}
%}
}
}

\institute[NC State University]
{%
Center for Geospatial Analytics
% \tiny
%\n{1}NCSU Department of Marine, Earth and Atmospheric Sciences \\
%\n{2}NCSU Center for Geospatial Analytics \\
%\n{3}U.S. Fish and Wildlife Service

\bigskip
\includegraphics[width=0.3\textwidth]{logos/ncstate}
}

\date{March 18, 2016}

\setbeamercovered{transparent}

\hypersetup{%
 pdfauthor={Vaclav Petras},%
 pdfsubject={GRASS GIS and lidar talk at US-IALE 2016},%
 pdfkeywords={UAV} {UAS} {point clouds} {lidar}
   {v.in.lidar} {r.in.lidar} {v.decimate} {v.out.lidar} {libLAS}
   {geospatial modeling} {GRASS GIS}
   {free software} {open source} {open science}
}

\usepackage{tipa}
\newcommand{\pron}[2]{#1 [#2]}


\newcommand{\beginbackup}{
  \newcounter{framenumbervorappendix}
  \setcounter{framenumbervorappendix}{\value{framenumber}}
}
\newcommand{\backupend}{
  \addtocounter{framenumbervorappendix}{-\value{framenumber}}
  \addtocounter{framenumber}{\value{framenumbervorappendix}}
}


%%%%%%%%%%%%%%%%%%%%%%%%%%%%%%%%%%%%%%%%%%%%%%%%%%%%%%%%%%%%%%%%%%%%
% when images are placed in these directories, we don't have to specify the directory
% just the filename
\graphicspath{{img/}{figures/}{images/}}


%%%%%%%%%%%%%%%%%%%%%%%%%%%%%%%%%%%%%%%%%%%%%%%%%%%%%%%%%%%%%%%%%%%%
%%%%%%%%%%%%%%%%%%%%%%%%%%%%%%%%%%%%%%%%%%%%%%%%%%%%%%%%%%%%%%%%%%%%
%%%%%%%%%%%%%%%%%%%%%%%%%%%%%%%%%%%%%%%%%%%%%%%%%%%%%%%%%%%%%%%%%%%%
%%%%%%%%%%%%%%%%%%%%%%%%%%%%%%%%%%%%%%%%%%%%%%%%%%%%%%%%%%%%%%%%%%%%
\begin{document}

\newcommand{\logowidth}{1.0em}
\newcommand{\logospace}{\hspace{0.2em}}
\newcommand{\includecclogo}[1]{\includegraphics[width=\logowidth]{./images/logos/#1}}

%%%%%%%%%%%%%%%%%%%%%%%%%%%%%%%%%%%%%%%%%%%%%%%%%%%%%%%%%%%%%%%%%%%%
\frame{
\titlepage
\begin{center}
\vspace{-3ex}
\href{http://creativecommons.org/licenses/by-sa/4.0/}{
\includecclogo{cc}
\logospace
\includecclogo{by}
\logospace
\includecclogo{sa}
}
\\
\footnotesize
available at\\
\href{http://wenzeslaus.github.io/grass-lidar-talks/}{\texttt{wenzeslaus.github.io/grass-lidar-talks}}
\end{center}
}


%%%%%%%%%%%%%%%%%%%%%%%%%%%%%%%%%%%%%%%%%%%%%%%%%%%%%%%%%%%%%%%%%%%%%
\begin{frame}{GRASS GIS}

\begin{columns}
\begin{column}{0.4\textwidth}

\begin{itemize}
  \item universal scientific and processing platform
  \begin{itemize}
  \item GUI, CLI, API
  \item from small laptops to supercomputers
  \end{itemize}
  \item lidar processing included
  \item data size and type challenges
\end{itemize}

\begin{center}
\includegraphics[width=0.3\textwidth]{logos/grass_gis}
\end{center}

\end{column}
\begin{column}{0.55\textwidth}

\begin{center}
  \includegraphics[width=\textwidth]{grass/count_and_modules}
\end{center}

\end{column}
\end{columns}

\end{frame}


%%%%%%%%%%%%%%%%%%%%%%%%%%%%%%%%%%%%%%%%%%%%%%%%%%%%%%%%%%%%%%%%%%%%%
\begin{frame}{Two core modules for lidar processing}


\begin{columns}
\begin{column}{0.5\textwidth}

\begin{itemize}
  \item \gmodule{r.in.lidar} -- binning and analyses, raster
  \item \gmodule{v.in.lidar} -- import of points, vector
\end{itemize}

\end{column}
\begin{column}{0.45\textwidth}

\begin{center}
  \includegraphics[width=0.8\textwidth]{grass/range_on_smooth_max_larger}

  \footnotesize
  r.in.lidar and r.neighbors, range smoothed maximum surface,
  551 million points from 90 files to 8076 x 7223 cells
  using 480MiB in 4 min
  % 1 + 37.262/60 + 1 + 47.132/60 + 11./60
\end{center}

% g.region -p
% north:      3885316.5
% south:      3877241.5
% west:       273832
% east:       281053
% nsres:      0.5
% ewres:      0.5
% rows:       16150
% cols:       14442
% cells:      233238300

% r.in.lidar max
% 550927433 points from 90 files found in region
% 920MiB
% 1m39.372s

% g.region -p
% north:      3885316.5
% south:      3877241.25
% west:       273831.75
% east:       281053.5
% nsres:      0.75
% ewres:      0.75
% rows:       10767
% cols:       9629
% cells:      103675443

% r.in.lidar max
% 550927433 points from 90 files found in region
% 430MiB
% 1m38.932s


% g.region -p
% north:      3885317
% south:      3877241
% west:       273831
% east:       281054
% nsres:      1
% ewres:      1
% rows:       8076
% cols:       7223
% cells:      58332948

% r.in.lidar max
% 550927433 points from 90 files found in region
% 250MiB
% 1m37.262s

% r.neighbors input=max size=7
% 11 sec

% r.in.lidar range
% 550927433 points from 90 files found in region
% 480MiB
% 1m47.132s

% r.in.lidar skewness
% 551004149 points from 90 files found in region
% 8.5 GiB
% 3m10.144s

% r.in.lidar mean class=2
% 28,669,253 points from 90 files found in region
% 480MiB
% 2 min 10 sec

% r.neighbors -c --overwrite input=ground_mean size=25
% 2 min 33 sec

\end{column}
\end{columns}

\end{frame}


%%%%%%%%%%%%%%%%%%%%%%%%%%%%%%%%%%%%%%%%%%%%%%%%%%%%%%%%%%%%%%%%%%%%%
\begin{frame}{Binning}

\begin{columns}
\begin{column}{0.4\textwidth}


 \begin{itemize}
  \item analyze with \gmodule{r.in.lidar}
  \item statistics of point counts, height and intensity
  \item using different resolutions
  \item produces raster from points
  \begin{itemize}
    \item n, min, max, sum
    \item mean, range, skewness, \ldots
  \end{itemize}
  \item limits import by
  \begin{itemize}
    \item range of Z
    \item return, class
  \end{itemize}
  \item multiple input files at once
  \item subsequent raster-based processing
\end{itemize}

\end{column}
\begin{column}{0.45\textwidth}

\begin{center}
  \includegraphics[width=0.75\textwidth]{grass/rinlidar_region}

  \footnotesize
  r.in.lidar, 578 million points in 90 files to 1882 $\times$ 1651 cells using 50MiB in 2 min

% time r.in.lidar file=list_las_files.txt -e method=n output=count res=5 --o
% 578628031 points from 90 files found in region
%
% r.in.lidar: 2m14.852s 50MiB
% r.info map=count@all -g
% north=3886000
% south=3876590
% east=281705
% west=273450
% nsres=5
% ewres=5
% rows=1882
% cols=1651
% cells=3107182
% datatype=CELL
% ncats=0
%
% r.in.lidar: 2m25.048s 1.2GiB
% r.info map=count_2@all -g
% north=3886000
% south=3876590.5
% east=281700.5
% west=273452
% nsres=0.5
% ewres=0.5
% rows=18819
% cols=16497
% cells=310457043
% datatype=CELL
% ncats=0

% g.region rast=count_2@all
% r.univar map=count_2@all -e
% 30 sec

\end{center}

\end{column}
\end{columns}

\end{frame}

%%%%%%%%%%%%%%%%%%%%%%%%%%%%%%%%%%%%%%%%%%%%%%%%%%%%%%%%%%%%%%%%%%%%%
\begin{frame}{Height above a surface}

\begin{columns}
\begin{column}{0.4\textwidth}

\begin{itemize}
  \item new feature in \gmodule{r.in.lidar}
\end{itemize}

\begin{center}
\includegraphics[width=\textwidth]{images/features/base_raster}
\end{center}

\bigskip
\flushright
\footnotesize
9 min, 480 MiB RAM, 2.7 GHz, SSD, Ubuntu
% (R 2150MB/s, W 1200MB/s)
\\
578 million points in 90 files

\end{column}
\begin{column}{0.55\textwidth}

\begin{center}
  \includegraphics[width=\textwidth]{grass/max_height_10m_on_ground_from_neighbors_smaller_area_top}
\end{center}

\end{column}
\end{columns}

\end{frame}


%%%%%%%%%%%%%%%%%%%%%%%%%%%%%%%%%%%%%%%%%%%%%%%%%%%%%%%%%%%%%%%%%%%%%
\begin{frame}{Binning into 3D raster}

\begin{columns}
\begin{column}{0.28\textwidth}

\begin{itemize}
  \item \gmodule{r3.in.lidar}
  \item generally similar to \gmodule{r.in.lidar}
  \item 3D raster
  \item proportional count
  \begin{itemize}
    \item count per 3D cell relative to the count per vertical column
  \end{itemize}
  \item intensity can be used instead of count
\end{itemize}

\bigskip
\footnotesize
under development

\end{column}
\begin{column}{0.7\textwidth}

\begin{center}
  \includegraphics[width=\textwidth]{grass/red_green_3d}
\end{center}

\end{column}
\end{columns}

% vis also possible using Tangible Landscape

\end{frame}


%%%%%%%%%%%%%%%%%%%%%%%%%%%%%%%%%%%%%%%%%%%%%%%%%%%%%%%%%%%%%%%%%%%%%
\begin{frame}{Decimation, interpolation}

\begin{columns}
\begin{column}{0.28\textwidth}


Create surface with \gmodule{v.in.lidar} and interpolation
 \begin{itemize}
  \item few/sparse points, smaller extent
%   \item import selected points (spatial extent, classes and returns)
  \item interpolate smooth surface without NULLs
  \end{itemize}

\begin{itemize}
  \item \gmodule{v.in.lidar}
  \item filtering same as in \gmodule{r.in.lidar}
  \item interpolation modules
  \item decimation (count-based as effective grid-based decimation important points)
\end{itemize}

\end{column}
\begin{column}{0.7\textwidth}

% TODO: UAV image

\begin{center}
  \includegraphics[width=\textwidth]{grass/range_on_ground_from_north}

  \footnotesize
  range from \gmodule{r.in.lidar} on ground obtained
  from \gmodule{v.in.lidar} followed by \gmodule{v.surf.rst}
\end{center}

\end{column}
\end{columns}

\end{frame}


%%%%%%%%%%%%%%%%%%%%%%%%%%%%%%%%%%%%%%%%%%%%%%%%%%%%%%%%%%%%%%%%%%%%%
\begin{frame}{Regional scale}

\begin{itemize}
  \item trade-off have enough memory to avoid using \texttt{percent} option
  \item no IDs (lim of int)
  \item \texttt{-t} do not create attribute table
  \item \texttt{-b} do not build topology (applicable to other modules as well)
  \item \texttt{-c} store only coordinates, no categories or IDs
  \item 64bit version
\end{itemize}

\end{frame}


%%%%%%%%%%%%%%%%%%%%%%%%%%%%%%%%%%%%%%%%%%%%%%%%%%%%%%%%%%%%%%%%%%%%%
\begin{frame}{Ground detection}

\begin{columns}
\begin{column}{0.28\textwidth}

\begin{itemize}
\item \gmodule{v.lidar.edgedetection}, \gmodule{v.lidar.growing}, \gmodule{v.lidar.correction}
\begin{itemize}
  \item uses returns
\end{itemize}

\item \amodule{v.lidar.mcc}
\begin{itemize}
  \item multiscale curvature based classification algorithm\footnotemark[1]
\end{itemize}

\end{itemize}

\end{column}
\begin{column}{0.7\textwidth}

\begin{center}
  \includegraphics[width=\textwidth]{grass/mcc_default}
\end{center}

\end{column}
\end{columns}

\bigskip

% cannot use normal footnotes because they are limited to the column
\footnoterule
\footnotesize
\footnotemark[1]
Evans, J. S. \& Hudak, A. T. 2007: A Multiscale Curvature Algorithm for Classifying Discrete Return LiDAR in Forested Environments.
% IEEE TRANSACTIONS ON GEOSCIENCE AND REMOTE SENSING 45(4): 1029 - 1038.

\end{frame}


%%%%%%%%%%%%%%%%%%%%%%%%%%%%%%%%%%%%%%%%%%%%%%%%%%%%%%%%%%%%%%%%%%%%%
\begin{frame}{Sky-view factor}

\begin{itemize}
  \item \amodule{r.skyview} (percentage of visible sky)
\end{itemize}

\begin{center}
  \includegraphics[width=0.45\textwidth]{vis/shade}
  ~
  \includegraphics[width=0.45\textwidth]{vis/skyview_ridges}
  \\
  {\footnotesize comparison of shaded relief and sky-view factor}
  % 412*4 (1648) x 490*4 (1960)
\end{center}

\end{frame}


%%%%%%%%%%%%%%%%%%%%%%%%%%%%%%%%%%%%%%%%%%%%%%%%%%%%%%%%%%%%%%%%%%%%%
\begin{frame}{Local relief model (LRM)}

\begin{itemize}
  \item \amodule{r.local.relief} (micro-topography, features other than trend)
\end{itemize}

\begin{center}
  \includegraphics[width=0.4\textwidth]{vis/elevation}
  ~~
  \includegraphics[width=0.4\textwidth]{vis/lrm}\\
  \footnotesize
  30-60cm wide, 30cm deep, 60m long gully (resolution 30cm)
  % 294 rows, 325 cols (88.2m x 97.5m)
\end{center}

\end{frame}


%%%%%%%%%%%%%%%%%%%%%%%%%%%%%%%%%%%%%%%%%%%%%%%%%%%%%%%%%%%%%%%%%%%%%
\begin{frame}{Analytical shading}

\begin{columns}
\begin{column}{0.35\textwidth}

\begin{itemize}
  \item \amodule{r.shaded.pca}
  \item relief shades from various directions
  \item PCA of shades
  \item combined into RGB composition
\end{itemize}

\footnotesize
%\footnotemark[1]
Devereux, B. J., Amable, G. S., \& Crow, P. P. (2008). Visualisation of LiDAR terrain models for archaeological feature detection. Antiquity, 82(316), 470-479.

\end{column}
\begin{column}{0.6\textwidth}

\begin{center}
  \includegraphics[width=\textwidth]{vis/pca_shade_road}
\end{center}

\end{column}
\end{columns}

\end{frame}


%%%%%%%%%%%%%%%%%%%%%%%%%%%%%%%%%%%%%%%%%%%%%%%%%%%%%%%%%%%%%%%%%%%%%
\begin{frame}{Landforms}

\begin{columns}
\begin{column}{0.3\textwidth}

\begin{itemize}
  \item \amodule{r.geomorphon}
  \item geomorphons - a new approach to classification of landform\footnotemark[1]
\end{itemize}

\footnotesize
\footnotemark[1]
Jasiewicz, J., Stepinski, T., 2013,
Geomorphons - a pattern recognition approach to classification and mapping of landforms, Geomorphology%
% vol. 182, 147-156 (DOI: 10.1016/j.geomorph.2012.11.005)

\end{column}
\begin{column}{0.68\textwidth}

\begin{center}
  \includegraphics[width=\textwidth]{vis/geomorphon_3d}
\end{center}

\end{column}
\end{columns}

\end{frame}

%%%%%%%%%%%%%%%%%%%%%%%%%%%%%%%%%%%%%%%%%%%%%%%%%%%%%%%%%%%%%%%%%%%%%
\begin{frame}{Integration with PDAL}

\begin{columns}
\begin{column}{0.5\textwidth}

\begin{block}{PDAL}
 \begin{itemize}
  \item formats besides LAS/LAZ
  \item algorithms, filters, decimations
 \end{itemize}
\end{block}

\begin{block}{Experimental integration}
 \begin{itemize}
  \item \module{v.in.pdal}
  \item reprojection during import
  \item ground filter
  \item compute height as a difference from ground
 \end{itemize}
\end{block}

\end{column}
\begin{column}{0.25\textwidth}

\begin{center}
  \includegraphics[width=\textwidth]{logos/pdal_bubbles}\\
  \includegraphics[width=\textwidth]{logos/pdal_text}
\end{center}

\end{column}
\end{columns}

\end{frame}


%%%%%%%%%%%%%%%%%%%%%%%%%%%%%%%%%%%%%%%%%%%%%%%%%%%%%%%%%%%%%%%%%%%%%
\begin{frame}{Acknowledgements}

\begin{block}{Software: GRASS GIS}
Presented functionality is work done by Vaclav Petras, Markus Metz, and the GRASS development team.
Thanks to users for feedback and testing, especially to
Doug Newcomb, Markus Neteler, Laura Belica, and William Hargrove.
\end{block}

\begin{block}{Datasets}
\footnotesize
Data for
\href{http://ncsu-osgeorel.github.io/uav-lidar-analytics-course/}%
  {GIS595/MEA792: UAV/lidar Data Analytics} course

\smallskip

Nantahala NF, NC: Forest Leaf Structure, Terrain and Hydrophysiology:
Lidar data acquisition and processing completed
by the National Center for Airborne Laser Mapping (\href{http://www.ncalm.org}{NCALM}).
NCALM funding provided by NSF's Division of Earth Sciences, Instrumentation and Facilities Program.
EAR-1043051.
Obtained from \href{http://www.opentopography.org/}{OpenTopography}.
\url{http://dx.doi.org/10.5069/G9HT2M76}
\end{block}

\end{frame}


%%%%%%%%%%%%%%%%%%%%%%%%%%%%%%%%%%%%%%%%%%%%%%%%%%%%%%%%%%%%%%%%%%%%%
\begin{frame}{}

% logo at the bottom can be moved down
\vspace*{0.05\textheight}

\begin{block}{Summary}
 \begin{itemize}
  \item rasterize early
  \item make use of existing methods for raster and vector processing
  \item 3D rasters, PDAL integration
  \item the plan for next 30 years driven by users
    -- \href{https://lists.osgeo.org/listinfo/grass-user}{grass-user mailing list}
 \end{itemize}
\end{block}

\bigskip
\centering

\begin{tabular}{clc}
\begin{minipage}{0.16\textwidth}
\includegraphics[width=\textwidth]{logos/grass_gis}
\end{minipage}
&
\begin{minipage}{0.4\textwidth}
\footnotesize
\href{https://grass.osgeo.org/download/}{%
Get GRASS GIS 7.1 development version at\\
\texttt{grass.osgeo.org/download}%
}

\bigskip

{
\footnotesize
Slides available at\\
\href{http://wenzeslaus.github.io/grass-lidar-talks/}%
  {\texttt{wenzeslaus.github.io/grass-lidar-talks}}
}
\end{minipage}
&
\begin{minipage}{0.2\textwidth}
\includegraphics[width=\textwidth]{talks_qr}
\end{minipage}
\end{tabular}

\end{frame}


\end{document}
